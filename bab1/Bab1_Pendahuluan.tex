\chapter{PENDAHULUAN}

\section{Latar Belakang}
\label{sec:1-LatarBelakang}

Pemanggilan referensi pertama kali dengan seluruh nama author \citep*{hinton2006fast} dan pemanggilan referensi untuk selanjutnya hanya nama author pertama dan et al \citep{hinton2006fast}. Daftar pustaka masukkan di \textbf{MainTemplateDisertasi.bib}.

\begin{figure}[H]
\centerline{\includegraphics[width=.6\textwidth]{bab1/dir_gambar/DeepSeek-Logo.png}}
\caption{Citra yang tidak sehat.}
\label{cfp_bab1}
\end{figure}

Pemanggilan label pada gambar, tabel, dan sejenisnya menggunakan Gambar \ref{cfp_bab1}. 

\section{Rumusan Masalah}
\label{sec:2-Rumusanmasalah}
Berdasarkan latar belakang yang telah disampaikan, dapat disimpulkan beberapa permasalahan yang dirumuskan sebagai berikut:

\begin{enumerate}
 \item Masalah 1
 \item Masalah 2
 \item Masalah 3
\end{enumerate}

\section{Tujuan dan Batasan Penelitian}
\subsection{Tujuan Penelitian}
\label{sec:3-TujuanPenelitian}
Tujuan dari penelitian ini dirangkum sebagai berikut:
\begin{enumerate}
    \item Tujuan satu.
    \item Tujuan dua.
    \item Tujuan tiga.
\end{enumerate}

\subsection{Batasan Penelitian}
\label{sec:3-BatasanPenelitian}
Untuk menghindari meluasnya permasalahan dalam domain yang diteliti, batasan penelitian dibuat dan diringkas dalam poin-poin berikut:
\begin{enumerate}
    \item Batasan 1.
    \item Batasan 2. 
    \item Batasan 3.
\end{enumerate}

\section{Sistematika Penulisan}
