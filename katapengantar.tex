\newpage %Acknowledgment
\addcontentsline{toc}{chapter}{KATA PENGANTAR}
\begin{center}
\begin{large}\textbf{KATA PENGANTAR}\\\end{large}
\end{center}
\vspace{5mm}
\textit{Bismillahirrohmaanirrohiim} \\
\textit{Assalamu'alaikum Warahmatullahi Wabarakatuh}

Alhamdulillahi rabbil 'aalamiin. Segala puji dan syukur hanya kepada ALLAH SWT, atas rahmat dan karunia-NYA, Penulis dapat menyelesaikan Tugas Akhir ini. Tugas akhir ini disusun sebagai salah satu syarat untuk memperoleh gelar Sarjana Komputer dari Universitas Gunadarma. Judul Tugas Akhir ini adalah "JUDUL TUGAS AKHIR". Shalawat dan salam Penulis sampaikan kepada suri teladan kita, manusia biasa dengan akhlak luar biasa, Rasulullah SAW.

Penulis menyadari bahwa dalam proses penyusunan Tugas Akhir ini, tidak terlepas dari bantuan, bimbingan, dorongan, dan doa yang tulus dari banyak pihak. Oleh karena itu, pada kesempatan yang berbahagia ini, dengan segala kerendahan dan ketulusan hati, penulis ingin mengucapkan terima kasih yang sebesar-besarnya kepada:


\begin{enumerate}
\item Prof. Dr. E.S. Margianti, S.E., MM.,  selaku Rektor Universitas Gunadarma.
\item Prof. Dr.-Ing. Adang Suhendra, SSi., SKom, MSc., selaku Dekan Fakultas Teknologi Industri Universitas Gunadarma.
\item Prof. Dr. Lintang Yuniar Banowosari, S.Kom., M.Sc., selaku Ketua Program Studi Informatika Universitas Gunadarma.
\item Dr. Edi Sukirman, SSi., MM., M.I. Kom., selaku Kepala Bagian Sidang Ujian Universitas Gunadarma.
\item Nama Dosen Pembimbing dengan Gelar, selaku Dosen Pembimbing yang telah banyak meluangkan waktu dalam membimbing, mengarahkan, memberi masukan, ilmu pengetahuan, dan koreksi dengan penuh kesabaran, sehingga tugas akhir ini menjadi lebih baik.
\item Kedua orang tua, yaitu Nama Ayah dan Nama Ibu sebagai orang tua penulis yang telah memberikan dukungan yang luar biasa beserta do'a yang tidak kunjung putus sehingga tugas akhir ini dapat selesai dengan hasil yang terbaik.
\item Seluruh rekan dan pihak yang telah membantu selama mengerjakan dan meyelesaikan tugas akhir ini.
\end{enumerate}

Akhir kata semoga tugas akhir ini bermanfaat bagi siapa saja yang mengkajinya dan dapat dikembangkan serta disempurnakan untuk menambah kemanfaatan bagi banyak orang.

\vspace{0.5 cm}

\textit{Wabillahi taufiq wal hidayah}

\textit{Wassalamu'alaikum Warahmatullahi Wabarakatuh}

% Bagian akhir dari kata pengantar

% \vspace{0.5 cm}

 %\textit{Wabillahi taufiq wal hidayah}

 %\textit{Wassalamu'alaikum Warahmatullahi Wabarakatuh}

\begin{flushleft}
Depok, 3 April 2024\\
Penulis
\vspace{1.5 cm}

(Nama Lengkap Penulis)

\end{flushleft} 
